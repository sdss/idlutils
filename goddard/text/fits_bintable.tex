\documentstyle[11pt]{article}

\setlength{\oddsidemargin}{0in}		%  1 inch (0 + 1) from left.
\setlength{\topmargin}{-0.5in}		%  1 inch (1.5 - 0.5) from top.
\setlength{\textwidth}{6.5in}		%  1 inch (8.5 - 1 - 6.5) from right.
\setlength{\textheight}{9in}		%  1 inch (11 - 1 - 9) from bottom.

\addtolength{\parskip}{0.5\baselineskip}

\begin{document}			% Remember to put in \end{document}.
\thispagestyle{empty}			% Don't number first page.

\begin{center}
{\Large\bf IDL Software for FITS Binary Tables}\\
\mbox{}\\
William Thompson\\
NASA Goddard Space Flight Center\\
2 July 1993\\
Modified W. Landsman 12 Aug 1997 \\
Modified C. Markwardt 10 Jan 2000
\end{center}

This software package was developed to support reading and writing FITS binary
table extensions.  The primary source for a description of the FITS binary
table extension is the document ``Binary Table Extension to FITS''
by W. D.  Cotton, D. Tody and W. Pence (1995, A\&AS, 113, 159).  This
information is also included  in the FITS standards document
``Definition of the Flexible Image Transport System (FITS)'' by Hanisch et al.
(2001, A\&A, 376, 359), and at the Web site \\
{\tt http://heasarc.gsfc.nasa.gov/docs/heasarc/fits.html}.  As well as routines
that are directly related to binary tables, there are also supporting routines
which allow the user to read and write primary FITS header and data units
(HDUs), and to copy FITS files to and from tapes.

\section{Primary FITS Header and Data Units}

Every FITS file must contain a primary Header and Data Unit (HDU).  This part
represents the most basic kind of FITS file.  It allows for the storage of an
array with NAXIS dimensions, but one can also have a primary HDU which only
consists of a header without any data (signaled in the header by setting
NAXIS=0).  All FITS extension HDUs, must then follow the primary HDU.

The following routines are for writing primary FITS header and data units.
They are listed in the order they would be used to create a basic FITS file to
which binary table extension could then be appended.
\begin{quote}
\begin{description}
\item[FXHMAKE:]
Creates the primary FITS header based on the array passed as an optional
parameter.  If no data array is passed, then a header array consistent with no
data in the primary HDU is created.  When appending FITS extensions, such as
binary tables, to the file, FXHMAKE should be called with the /EXTEND qualifier
to generate a header record that reads ``EXTEND=T'' (true).
\item[FXADDPAR:]
Adds keyword=value records to the primary header, or modifies those that are
already there.  The use of this routine is entirely at the user's
discretion---the required keywords are inserted by FXHMAKE.
\item[FXWRITE:]
Writes the primary header and (optional) data array to a FITS file on disk.
\end{description}
\end{quote}

There are a number of routines available to read and interpret FITS primary
HDUs.  For example, the IDL Astronomy User's Library, available at \\
{\tt http://idlastro.gsfc.nasa.gov/homepage.html}, contains a number of routines including
READFITS\@.  The following lists two routines included in this package to
support this function.
\begin{quote}
\begin{description}
\item[FXREAD:]
Reads a basic FITS file, or the primary HDU from a FITS file with extensions.
This routine can either read the entire array, or a sub-array.  It also gives
the user the option of reducing the amount of data read either by reading every
$n$th pixel, or by averaging together \mbox{$n \times n$} pixels.
\item[FXPAR:]
Extracts values from FITS headers.
\end{description}
\end{quote}

\section{Creating FITS binary table files}
\label{creating}

The processes of creating a FITS binary table file is fairly easy.  However,
there are enough individual steps involved in creating the file, that it is
anticipated that it will be carried out primarily using procedure files, rather
than interactively from the keyboard.

The routines used in writing FITS binary tables files are:
\begin{quote}
\begin{description}
\item[FXBHMAKE:]
Starts the definition of the FITS binary table header.  At this time the number
of rows in the table is defined, but no information about the individual
columns is as yet included.  Optionally, the extension is given a name.
\item[FXBADDCOL:]
Add the information about the columns to the binary table extension header.
This routine is called once for each column that will be in the table.
\item[FXBCREATE:]
Opens the FITS file that the binary table will be appended to and writes the
header.
\item[FXBWRITE or FXBWRITM:]
Writes the data into the binary table.  A separate call to FXBWRITE is
made for every row and and every column in the table.  FXBWRITM may be
used to write multiple rows and columns in a single call.
\item[FXBFINISH:]
Finishes up and closes the file containing the newly created binary table.
\end{description}
\end{quote}

The basic steps involved in creating a FITS binary table extension file are as
follows:
%
\begin{itemize}
\item
	First the primary FITS data unit must be created.  At the very minimum,
	this will include a FITS header with the keyword ``EXTEND'' set to
	T(rue).  It may also have data associated with it, or it may simply
	have ``NAXIS'' set to zero to signal that there is no primary data
	array.  This step is carried out through the FXHMAKE and FXWRITE
	routines.
\item
	Next, the binary table extension header must be created, and the
	various columns to be used have to be defined.  The routines FXBHMAKE
	and FXBADDCOL take care of this.
\item
	Then, the extension file must be opened, and the header written out.
	FXBCREATE takes care of this.
\item
	The next step is to actually write the data arrays themselves into the
	table.  This is done using multiple calls to FXBWRITE or one or more
	calls to FXBWRITM.
\item
	Finally, the table file is closed with the FXBFINISH command.
\end{itemize}

The following IDL statements demonstrate how to use these routines to create a
simple binary table with three columns and five rows.  Some test arrays are
generated to write into these columns, and are slightly modified for each row,
to make the test more complete.
%
\begin{quote}
\begin{verbatim}
;  Create the data to write to the binary table.
;
a = intarr(10,100)
b = reverse(intarr(20,100),2) - 100
c = fix(dist(50))
;
;  Create a primary header and write it out.
;
fxhmake,header,/initialize,/extend,/date
fxwrite,'sample.fits',header
;
;  Create a binary table extension header for a table with 5 rows.
;
fxbhmake,header,5,'TESTEXT','Test binary table extension'
;
;  Create the columns for the a, b, and c arrays.
;
fxbaddcol,acol,header,a,'Column 1'
fxbaddcol,bcol,header,b,'Column 2'
fxbaddcol,ccol,header,c,'Column 3'
;
;  Write out the extension header.
;
fxbcreate,unit,'sample.fits',header
;
;  Write out the data.  For each row, multiply the test arrays by the row
;  number.
;
for i=1,5 do fxbwrite,unit,a*i,acol,i
for i=1,5 do fxbwrite,unit,b*i,bcol,i
for i=1,5 do fxbwrite,unit,c*i,ccol,i
;
;  Same result is achieved with following call to FXBWRITM:
;  x = findgen(5) + 1
;  fxbwritm, unit, [acol, bcol, ccol], a*x, b*x, c*x
;
;  Close the binary extension.
;
fxbfinish,unit
end
\end{verbatim}
\end{quote}
%
The primary FITS header created by this routine is very simple, and looks like
this
%
\begin{quote}
\begin{verbatim}
SIMPLE  =                    T /Written by IDL:  Sun Mar 28 20:24:37 2010
BITPIX  =                    8 /
NAXIS   =                    0 /
EXTEND  =                    T /File contains extensions
DATE    = '2010-03-29'         /
END
\end{verbatim}
\end{quote}
%
And the binary table extension header looks like this
%
\begin{quote}
\begin{verbatim}
XTENSION= 'BINTABLE'           /Written by IDL:  30-Jan-1992 11:35:49.00
BITPIX  =                    8 /
NAXIS   =                    2 /Binary table
NAXIS1  =                11000 /Number of bytes per row
NAXIS2  =                    5 /Number of rows
PCOUNT  =                    0 /Random parameter count
GCOUNT  =                    1 /Group count
TFIELDS =                    3 /Number of columns
EXTNAME = 'TESTEXT '           /Test binary table extension
TFORM1  = '1000I   '           /Integer*2 (short integer)
TTYPE1  = 'Column 1'           /Label for column 1
TDIM1   = '(10,100)'           /Array dimensions for column 1
TFORM2  = '2000I   '           /Integer*2 (short integer)
TTYPE2  = 'Column 2'           /Label for column 2
TDIM2   = '(20,100)'           /Array dimensions for column 2
TFORM3  = '2500I   '           /Integer*2 (short integer)
TTYPE3  = 'Column 3'           /Label for column 3
TDIM3   = '(50,50) '           /Array dimensions for column 3
END
\end{verbatim}
\end{quote}

One thing that sometimes confuses first time users is that UNIT is an output
parameter of FXBCREATE, not an input parameter.  One does not specify the
logical unit number for the file; instead FXBCREATE assigns one via a call to
GET\_LUN.  A similar situation exists in regard to the column number argument
to FXBADDCOL.  Column numbers are assigned in the order in which FXBADDCOL is
called.  In the above example, the values of acol, bcol, and ccol will be
returned by FXBADDCOL as 1, 2, and 3 respectively.

\section{Reading FITS binary table files}

The process of reading FITS binary tables is much simpler than writing them,
since the structure of the table is already set.  There are only three basic
steps in reading a FITS binary table file:
%
\begin{itemize}
\item
	The file is opened, and the binary table extension selected, with the
	routine FXBOPEN.
\item
	Data is read from the table with the routines FXBREAD or
	FXBREADM.  Using FXBREAD, a particular row and column can be
	read, or an entire column can be read into a single array
	(except for columns containing variable-length arrays).
	FXBREADM can read multiple columns in a single procedure call.
\item
	The file is closed with the routine FXBCLOSE.
\end{itemize}
%
For instance, the binary table created in the above example could be read with
the following statements.
%
\begin{quote}
\begin{verbatim}
IDL> FXBOPEN,UNIT,'sample.fits','testext',header
IDL> FXBREAD,UNIT,A,'Column 1'
IDL> FXBREAD,UNIT,B,'Column 2'
IDL> FXBREAD,UNIT,C,'Column 3'
IDL> FXBCLOSE,UNIT
IDL> HELP,A,B,C
A               INT       = Array(10, 100, 5)
B               INT       = Array(20, 100, 5)
C               INT       = Array(50, 50, 5)
\end{verbatim}
\end{quote}
%
In the above example the columns were read individually using three
calls to FXBREAD.  The same effect can be achieved by calling FXBREADM
in the following way.
%
\begin{quote}
\begin{verbatim}
IDL> FXBREADM,UNIT, ['Column 1', 'Column 2', 'Column 3'], A, B, C
\end{verbatim}
\end{quote}
%
Note that, because the entire columns were read in, the arrays A, B, and C each
have an extra last dimension of 5.  Also, the same comment in
Section~\ref{creating} above about FXBCREATE and UNIT also applies to FXBOPEN.

The following routines support reading FITS binary table extensions:
\begin{quote}
\begin{description}
\item[FXBOPEN:]
Opens a FITS binary table extension for reading.  One can open several binary
tables at once, either in the same or different files, by specifying different
variable names for FXBOPEN to store the logical unit number into.
\item[FXBREAD or FXBREADM:]
Reads data from a column in a FITS binary table.  One can read an entire
column, a range of rows within a column, or a single row and column
combination.  FXBREADM can read multiple columns in a single call.
\item[FXBTDIM:]
Parses keywords from binary tables with a TDIM-like format.  See
Section~\ref{tdim} for more information.
\item[FXBHELP:]
Prints to the screen a simple table giving information about each of the
columns in the binary table.
\item[FXBFIND:]
Finds the values of keywords in the header associated with the binary table
columns.  For example, the command
\begin{quote}
\begin{verbatim}
fxbfind, header, 'TTYPE', columns, values, n_found
\end{verbatim}
\end{quote}
would return an array containing the values of the keywords TTYPE1, TTYPE2,
etc., the columns for which they were found, and how many were found.
\item[FXBCOLNUM:]
Returns the column number of a FITS binary table specified either as a number
or by name.
\item[FXBHEADER:]
Returns the header of a FITS binary table opened with FXBOPEN.  (Note that the
header can also be returned as an optional parameter of FXBOPEN.)
\item[FXBISOPEN:]
Returns whether or not a logical unit number points to a FITS binary table that
is open for read.
\item[FXBSTATE:]
Similar to FXBISOPEN, but returns a state variable denoting whether or not a
logical unit number points to an open FITS binary table, and if so whether that
table is open for read or for write.
\item[FXBCLOSE:]
Closes a FITS binary table that had been opened with FXBOPEN.
\end{description}
\end{quote}

\section{Modifying Existing FITS Binary Tables}

After a FITS binary table extension has been written, a subsequent
need may arise to modify or augment the table.  This may be required
if, for example, new data or corrections must be overwritten in a
pre-existing data file.  In general this is accomplished using the
standard procedures to write data to FITS columns, typically FXBWRITE
or FXBWRITM.  However, after a table extension has been created, it is
not possible to change its number of columns or the number of header
keywords.

Since the table has already been created, there is no need to call
FXBHMAKE or FXBADDCOL again.  An existing FITS table extension is
opened for write access using the FXBOPEN procedure with the ACCESS
keyword.
%
\begin{quote}
\begin{verbatim}
IDL> FXBOPEN,UNIT,'sample.fits','testext',header, ACCESS='RW'
\end{verbatim}
\end{quote}
%
The keyword value \verb|'RW'| indicates that the FITS file is both
readable and writable.  [ The default is \verb|'R'|, read-only. ]

After the table has be opened for writing, the FXBWRITE and/or
FXBWRITM procedures should be used to modify the desired columns.
Finally, FXBFINISH should be used to close the table.

\section{Enlarging an Existing FITS Binary Table}

After a FITS binary table has been created it may also be necessary to
increase the number of rows in the table.  As mentioned above, one
cannot change the number of columns.  The number of rows can be
changed to the new value \verb|NEW_NROWS| by using the following
FXBGROW procedure call.  .  FXBGROW is safe for variable-length tables.
%
\begin{quote}
\begin{verbatim}
IDL> FXBOPEN,UNIT,'sample.fits',1,access='RW'
IDL> FXBGROW,UNIT,HEADER, NEW_NROWS
IDL> FXFINISH,UNIT
\end{verbatim}
\end{quote}
%
Both the header and the unit must be passed to FXBGROW because both
disk and memory images of the header must be modified.  Also, be aware
that the number of rows in the table can never decrease by calling
FXBGROW.


\section{Multidimensional Array Facility}
\label{tdim}

The routines described here provide direct support for the Multidimensional
Array Facility described in C\&T\@.  This convention uses keywords TDIM$n$ of
the format
\begin{quote}
TDIM$n$ = '($D_1$, $D_2$, \ldots)'
\end{quote}
to define the dimensions associated with column $n$.  For an example of the use
of this keyword, see the sample binary table header in Section~\ref{creating}
above.

Support for this convention is automatic---FXBADDCOL inserts TDIM$n$ keywords
into the header, and FXBOPEN interprets any found in the header---unless the
/NO\_TDIM keyword is used.  Values of TDIM$n$ can also be overridden with the
DIMENSIONS keyword in the FXBREAD routine.

In addition to the keywords described in the binary tables extension proposal,
several additional keywords are supported by FXBADDCOL.  These keywords have a
one-to-one correspondence with standard keywords used in primary FITS headers,
i.e.
\begin{center}
\begin{tabular}{cc}
Additional Keyword	& Standard Equivalent\\
\hline
			&	\\
TDMIN$n$		& DATAMIN \\
TDMAX$n$		& DATAMAX \\
			&	\\
TDESC$n$		& CTYPE$m$ \\
TROTA$n$		& CROTA$m$ \\
TRPIX$n$		& CRPIX$m$ \\
TRVAL$n$		& CRVAL$m$ \\
TDELT$n$		& CDELT$m$ \\
\end{tabular}
\end{center}
The anticipated use of these keywords is such that TDMIN$n$ and TDMAX$n$ would
have a ordinary FITS record format, no different from their standard
equivalents, and that the rest would have a format similar to TDIM$n$.

\section{Variable-Length Array Facility}

These routines also support the Variable-Length Array Facility described in
C\&T\@.  Variable-length array columns are defined by using FXBADDCOL with the
/VARIABLE keyword.  Other than that, support for variable-length arrays is
automatic.  Some operations, such as reading entire columns, and the
multidimensional array facility described above, are not allowed with
variable-length arrays.

Ordinarily, the default THEAP value \mbox{(NAXIS1 $\times$ NAXIS2)} is used to
write the variable-length arrays.  However, a different THEAP value can be used
by using FXADDPAR to insert the desired value into the binary table header
before calling FXBCREATE.

\section{IEEE Not-a-Number (NaN) Special Values}

Data dropout in FITS binary table arrays are signalled in one of two ways.
Dropouts in integer arrays are signalled with values specified by TNULL$n$
keywords.  However, dropouts in floating point arrays (including single or
double precision, and real or complex) are signalled with standard IEEE NaN
(not-a-number) special values.  The routine FXBREAD will optionally translate
these NaN numbers into a user-specified value, given by the NANVALUE keyword.
Conversely, the same keyword, when used with the FXWRITE or FXBWRITE routines,
will write out NaN for any points in the array with the value of the NANVALUE
keyword.

\section{Bit, Logical, and Double-precision Complex Arrays}



Bit arrays (type ``X'' in FITS binary tables) are treated in IDL as byte arrays
with approximately 1/8th the number of elements.  Support for this is automatic
when reading binary tables.  Columns can be defined as type ``X'' when writing
binary tables if the BIT keyword is passed to FXBADDCOL giving the number of
bits, and the data array is of type byte.  Dimension information is ignored for
bit arrays, since the dimensions apply to the bits, and not to the bytes that
IDL processes.

Logical arrays (type ``L'' in FITS binary tables) are treated in IDL as byte
arrays.  Support for this is automatic when reading binary tables.  Columns can
be defined as type ``L'' when writing binary tables if the /LOGICAL keyword is
passed to FXBADDCOL, and the data array is of type byte.

\section{Virtual Columns}

It is possible to treat keywords in binary table headers as if they were
columns in the table, with the same value replicated for every row.  This
virtual column convention allows the user to have a unified view in a table
regardless of whether the information is stored in a table column and thus
capable of varying from row to row, or stored in the header and thus the same
for every row.

To use the virtual column convention, the user must call FXBREAD with the
/VIRTUAL keyword, and must also reference the desired information by name
rather than by column number.  FXBREAD will then look first for a column with
that name.  If it doesn't find one, it then looks for a keyword with that name
in the header.

\section{Associated routines}

The remaining routines are mainly used internally by the other routines
mentioned above.
\begin{quote}
\begin{description}
\item[FXHCLEAN:]
Remove obsolete keywords---called by FXHMAKE, FXHBMAKE.
\item[FXPARPOS:]
Find position in FITS header---called by FXADDPAR.
\item[FXBFINDLUN:]
Find LUN in FXBINTABLE---called by FXBCREATE, FXBOPEN.
\item[FXBPARSE:]
Parse binary table header---called by FXBCREATE, FXBOPEN.
\item[FXBTFORM:]
Parse TFORM column descriptor---called by FXBPARSE.
\item[FXHREAD:]
Read FITS header---called by FXBOPEN.
\item[FXFINDEND:]
Find the last FITS record---called by FXBCREATE.
\item[WHERENAN:]
Find points equal to IEEE NaN---called by FXBREAD.
\item[BOOST\_ARRAY:]
Resize array, and append another array.
\item[STORE\_ARRAY:]
Resize array, and insert another array.
\item[DETABIFY:]
Removes tabs from strings.
\item[PRODUCT:]
Calculates total product of all elements of an array.
\item[FXTPIO\_READ:]
Reads a file from a tape---called from FXTAPEREAD.
\item[FXTPIO\_WRITE:]
Writes a file to a tape---called from FXTAPEWRITE.
\end{description}
\end{quote}

\section{Implementation notes}

The routines in this directory also make use internally of other routines from
the SDAS, FITS, and MISC directories from the Astronomy User's Library.  In
some cases these files are distributed along with the routines described here.

The file ``fxbintable.pro'' is an include file containing the definition of the
IDL common block FXBINTABLE.  This file must be in one of the directories
pointed to by the IDL search path parameter !PATH\@.  Normally, this is ensured
by keeping this file in the same directory with the IDL procedures found here.

\end{document}
